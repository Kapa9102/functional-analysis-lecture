% lec2.tex 
\lecture{2}{08:07 AM Tue, Sep 30 2025}{} 

\begin{definition}[]
Let $E $ be a set, and $d $ be a metric 
on $E $, we said that $(E, d)$ is complete if and only if
every cauchy sequence is convergent
\end{definition}
\begin{definition}[]
  A Banach space is a complete vectorial normed space
\end{definition}
\begin{example}
  \begin{itemize}
    \item $\KK (\RR , \CC )  $ is a Banach space
      \item $\KK^n  $  is a Banach space
        \item if $E $ is a normed space with finite dimension,
          then $E $ is Banach
          \item $\ell^{p}(\NN, \KK)  $  with $1 \leq p \leq +\infty  $ is Banach space ( Exercise ).
            \item 
              Every finite set product of Banach spaces is a Banach space too.
  \end{itemize}
\end{example}
\begin{proposition}
  The limit of a sequence in a Banach space (if it exists) is unique.
\end{proposition}
\begin{proposition}[]
Let $F $ be a closed subset of $E $. if $(x_{n})_{n \in  \NN} \subset F  $ is convergent to $x \in E $, 
then $x \in F $. 
\end{proposition}
\begin{center}
  \textcolor{purple} {
  \underline{\textsc{Remark:}} }
\end{center}
The importance of the notion of equivallent norms on a normed vector space is: \\
Let $\| . \| _{1} $ and $\| . \| _{2} $ be two equivallent norms on $V = E$. one has: 
\[
  (x_{n}) _{n \in  \NN} \text{ CV for }  \| . \| _{1} \iff 
  (x_{n}) _{n \in  \NN} \text{ CV for } \| . \| _{2}
\]
\begin{proposition}[Definition]
Let $V $ and $W $ be two norms on $V $ ( resp. $W $ ), and let $ f : V \longrightarrow W $ be a function.
the following statments are equivallent:
\begin{enumerate}
\item $f $ is continuous from $V $ into $W $ 
 \item and: 
   \[
   \forall x_0 \in  V, \forall \veps > 0, \exists \dd > 0, \forall x \in V:
   \quad \| x-x_0 \| \leq \dd \implies 
   \| f(x) -f(x_0)  \| \leq \veps 
   \]
  \item and:
    \[
    \forall x \in V, \forall (x_{n}) _{n \in \NN} \in V: \quad 
    x_{n} \rightarrow x \implies 
    f(x_{n}) \rightarrow f(x) ( \| f(x_{n} - f(x) )  \| \rightarrow 0 \quad n \rightarrow \infty )  
    \]
\end{enumerate}
\end{proposition}
\begin{center}
  \textcolor{purple} {
  \underline{\textsc{Remark:}} }
\end{center}
On finite dimension vector space, all the norm are equivallents.
\section{Duality}
Let $V $ and $W $ be two normed vector be two normed vector spaces
\begin{proposition}[]
Let $ A : V \longrightarrow W $ be a linear function ( Map ). The following are equivallent:
\begin{enumerate}
\item $A $ is continuous on $V $ 
  \item $A $ is continuous at $0 $ 
    \item $\exists c > 0$: $\| A(v)  \| \leq  c \| v \| \quad \forall v \in  V $  
\end{enumerate}
\end{proposition}
\begin{definition}[]
We denote by $L(V, W) $ be the set of all linear functions, we denote $\mathcal{L} (V, W)  $ the set
of all continuous linear functions
\end{definition}
\begin{proposition}[]
$L(V,W)  $ and $\mathcal{L} (V, W)  $ are vector spaces
\end{proposition}
\begin{proposition}[]
Let $V $ be a normed space, and $W $ be a Banach space then $\mathcal{L} (V, W)$ 
is Banach space.
\end{proposition}
\begin{definition}[]
Let $(x_{n})_{n \in \NN}  $ be sequence in a normed vector space $E $.
\begin{itemize}
  \item we say that the series $\sum_{n=1}^{\infty} x_n  $ converges if the sequence of partial sums 
    $s_{n} = \sum_{k=1}^{n} x_{k} $ converges.
    \item we say that the series is normally (absolutely) convergent 
      if the series of real positive numbers $\sum_{n=1}^{\infty} \| x_n  \|  $ converges
\end{itemize}
\end{definition}
\begin{proposition}[]
Let $E $ be a normed vector space. The following are equivallents:
\begin{enumerate}
\item $E $ is Banach.
  \item  Every normally convergent series is convergent.
\end{enumerate}
\end{proposition}
\begin{definition}[]
Let $E $ and $F $ be two normed vector spaces, A function (mapping) from $E $ into $F $ is called
linear isomorphism if its linear, continuous bijective and its inverse is continuous (naturally linear)
\end{definition}
We denote by ISO$(E, F)$ the set of all linear isomporphisms mapping from $E $ into $F $. 
\begin{theorem}[]
Let $E $ and $F $ be Banach spaces, then ISO$(E, F)$ is an open subset of $\mathcal{L} (E, F)  $  and the
map $  u \longrightarrow  u^{-1}$ from ISO$(E, F)$ into ISO$(F, E)  $ is continuous. More, one has 
if $u_0 \in \text{ISO} (E, F) $ and $u \in B(u_0, \frac{1}{\| u_0^{-1} \| })  $, then 
$u \in \text{ISO} (E, F) $, and:
\[
u^{-1} = \sum_{n=1}^{\infty} (id_{E} - u_0^{-1}u) ^n \cdot u_0^{-1}
\]
\end{theorem}
\begin{corollary}[Von Neumann]
  Let $E $ be a Banach space, and $u \in \mathcal{L} (E) = \mathcal{L} (E, E)  $. \\
  if $\| u \| <  1 $, then $id_{E} - u$ belongs to $\text{ISO} (E, E) = \text{ISO} (E) $ and it's 
  inverse is given by: 
  \[
    (Id_{E} - u) ^{-1} = \sum_{n=1}^{\infty} u^n 
  \]
\end{corollary}
  \textcolor{purple} {
  \underline{\textsc{Notations:}} } $E' = \mathcal{L} (E)  $, 
  and $E^{*} = L(E)  $.
  \begin{enumerate}
  \item $E'$ toplogical dual space of $E$. 
  \item $E^{*} $ algebraic dual space of $E$. 
  \end{enumerate}
  \begin{proposition}[]
  Let $E $ be a Banach space $E_0 $ be normed vector space.
  let $ p : E \longrightarrow E_0 $ a continuous linear surjection mapping. If there exists
  $c > 0 $ such that $\forall y_0 \in  E_0, \exists x \in  E $ such that $f(x) = y_0 $ and 
  $\| x \| \leq c \| y_0 \|  $ then $E_0 $ is Banach.
  \end{proposition}
  \begin{theorem}[]
  Let $E $ be a Banach space and let $F $ be a closed subspace of $E $. Then $E_{|F} $ is Banach too.
  \end{theorem}
  \begin{proposition}[]
  Let $A \subset E $. A is compact if from every sequence of elements of $A $, we can extract
  a subsequence converging to an element in $A $.
  \end{proposition}
  \begin{theorem}[Riesz]
    The closed unit ball is compact if and only if dim $E  < +\infty $, 
  \end{theorem}
% end of file
